\documentclass[12pt]{report}
\usepackage{listings}
\usepackage{underscore}
\usepackage{vhistory}
\usepackage{setspace}
\usepackage[bookmarks=true]{hyperref}
\hypersetup{
    bookmarks=false,    % show bookmarks bar?
    pdftitle={Software Requirement Specification},    % title
    pdfauthor={Yiannis Lazarides},                     % author
    pdfsubject={TeX and LaTeX},                        % subject of the document
    pdfkeywords={TeX, LaTeX, graphics, images}, % list of keywords
    colorlinks=true,       % false: boxed links; true: colored links
    linkcolor=blue,       % color of internal links
    citecolor=black,       % color of links to bibliography
    filecolor=black,        % color of file links
    urlcolor=purple,        % color of external links
    linktoc=page            % only page is linked
}%

\usepackage{geometry}
\geometry{margin=1.5in}
\onehalfspacing


\usepackage{atbegshi}% http://ctan.org/pkg/atbegshi
\AtBeginDocument{\AtBeginShipoutNext{\AtBeginShipoutDiscard}}


\title{%




\Huge{Követelményelemzés}\\
\vspace{2cm}

Marsjáró jármű\\
\vspace{2cm}

\LARGE{Verzió 1.0}
\vspace{2cm}

\small{
Készítették:\\
Cseresnyés Kristóf: @aPisC\\
Fikó Róbert: @robertfiko\\
Gőgös Márton: @Marzyh\\
Hosek Henrietta: @hosekhenrietta\\
Kurucz Ádám: @engreyight\\
Lippa Katalin: @katilippa\\
Miskolczi Péter: @BreadKingGod\\
Monhor Hubert: @dopamean\\
Novák-Schwartz József: @JNSchwartz\\
Szabó Tihamér: @SzaboTihi14\\
Szerednik László Tamás: @szeredniklaszlo\\
}
}
\date{}
\usepackage{hyperref}



\begin{document}

\maketitle

\tableofcontents
\begin{versionhistory}
  \vhEntry{1.0}{22.10.23}{Novák-Schwartz József}{1st revision}
  \vhEntry{1.1}{22.11.06}{Robert Fiko}{Subsystems, risk analysis and formatting}
  \vhEntry{1.2}{22.11.09}{BreadKingGod}{Meghajtas}
  \vhEntry{1.3}{22.11.11}{szerendiklaszlo}{Navigation, sensors}
  \vhEntry{1.4}{22.11.12}{engreyight}{Drill}
  \vhEntry{1.5}{22.11.12}{szabotihi}{Supply, food, drink}
  \vhEntry{1.6}{22.11.12}{Novák-Schwartz József}{Spacesuit}
  \vhEntry{1.6.1}{22.11.13}{Robert Fiko}{Merge and review}



  

\end{versionhistory}
\chapter{Projekt bemutatása}
Mars projekt
A NASA a Mars emberekkel történő felfedezéséhez magáncégek pályázatát várja a szükséges hardver és szoftverelemek kifejlesztéséhez az alábbi témakörökben:
\begin{enumerate}
  \item Marsbázis
  \item Marsjáró jármű
  \item Ellátmányűrhajó
  \item Mars-járó jármű
\end{enumerate}
Készítsen specifikációt 2 Mars-járó járműre az alábbiakat figyelembe véve:
\begin{itemize}
  \item A járműveknek maximum 4 ember ellátását kell biztosítaniuk a Mars-járó missziók
időtartamára
  \item A jármű élettartama minimum 5 év legyen a Marsi körülmények között
  \item A járműnek lehetőséget kell biztosítania az űrhajósok kiszállására és beszállására
közvetlenül a Marsra vagy a marsi bázisra
  \item A jármű képes legyen a misszióhoz szükséges segédfelszerelések szállítására, vizsgálatok
elvégzésére, valamint a begyűjtött kőzetminták szállítására
  \item Az eszközhöz csatlakozzon egy repülni képes, felderítő drón
  \item A járműnek képesnek kell lennie kommunikálnia a Marsbázissal és a Mars körül keringő
átjátszó műholdakkal valamint űrállomással
  \item Végezzen kockázatanalízist és a szükséges alrendszereknél alkalmazzon redundanciát
  \item A főbb alrendszereknek javíthatónak és cserélhetőnek kell lennie az ellátmány bázis
alkatrészei alapján
\end{itemize}
\chapter{Követelmények}
\section{Megvalósítás lépései}
\begin{itemize}
  \item Földi tervezés.
  \item Földi prototipizálás.
  \item Földi tesztelés. (Antarktisz és vulkán.)
  \item Holdi tesztelés.
  \item Marsi tesztelés?
  \item Marsi szállítás.
\end{itemize}
\section{Mérnöki követelmények}
\begin{itemize}
  \item A marsjáró mérete x*y*z.
  \item A marsjáró kereke (?).
  \item A marsi körülményeknek ellenálló anyag ?titánium?, ebböl készül a külsö ?n? réteg.
\end{itemize}

\subsection{A jármű alrendszerei\label{subsytems}}

A biztonságos és megbízható működés, illetve a modularitás elveinek betartása végett a 
járművet alrendszerekre bontjuk.

\subsubsection{Navigáció és szenzorok}
A navigációs alrendszer célja a vezetőt ellátni információkkal a terepviszonyokról, útvonalat tervezni a marsi felszínen, illetve egy bizonyos szintű önvezetést megvalósítani.

Szükséges moduláris érzékelők és eszközök:
\begin{itemize}
  \item műholdas antennák
  \item giroszkóp és gyorsulásmérő
  \item motorhő-érzékelő
  \item több irányba tekintő ultrahangos szenzorok és közelség- és mélységérzékelők
  \item 360 fokos és 3D térlátást biztosító nagy felbontású kamerák 
  \item számítógépes egység (az adatok feldolgozásához)
\end{itemize}

Biztosított funkciók:
\begin{itemize}
  \item helyzetmeghatározás: antennák által az űrbázis marsi műholdjaira csatlakozva
  \item térkép felépítése: műholddal készült felvételekkel, marsjáró által már bejárt helyekről készült adatrögzítés által
  \item környezetfelismerés: akadálymentes meghajtás szenzorok és kamerák által \begin {itemize}
      \item lehetséges-e a kívánt dőlésszögben való haladás
      \item megléphetetlen akadályok detektálása
    \end{itemize}
\end{itemize}

Különleges irányelvek:
\begin{itemize}
  \item alacsony üzemanyagszint: energiahasználat csökkentése a nem szükséges szenzorok kikapcsolásával, ezáltal növelni az esélyét a bázisra visszatérésnek
  \item útvonalon be nem tervezett akadály: újratervezés biztonságos útvonalat keresve
\end{itemize}

Becslések, gyártás és beszerzés, valamint ezek ledelegálása:
\begin{itemize}
  \item antennák, kamerák és szenzorok beszerzése: \begin{itemize}
      \item megfelelő mennyiségű éles- és tartalékkészlet megrendelése
      \item kapcsolatfelvétel az elvárt minőséget biztosító, és minél kedvezőbb árat kiszabó beszállítókkal
      \item időbecslés: 1.5 hónap
      \item pénzbecslés: 350 AK
    \end{itemize}
  \item navigációs szoftver: \begin{itemize}
      \item előfeltétel: navigációs szenzorok beszerzése, programot elkészítő csapat megbízása
      \item időigény: 6 hónap
      \item pénzbecslés: 10000 AK
    \end{itemize}
  \item összesített becslés: \begin{itemize}
      \item előfeltétel: navigációs szenzorok beszerzése, programot elkészítő csapat megbízása
      \item időigény: 7.5 hónap
      \item pénzbecslés: 10350 AK
    \end{itemize}
\end{itemize}

Ahol 1 Arany Krajcár (AK) = 1 EUR

\subsubsection{A fúró és szenzorai, vizsgálatok}

A marsjárónak több különböző fúrófejjel rendelkeznie, melyek cserélhetők. Ezeket használja egyrészt 30cm mély lyukak fúrására, illetve a felszín lekaparására. Az így keletkezett lyukak alján ezután párologtatást végez, és az így kiváló anyagokat összegyűjti. Emellett képes a felszínen már jelenlévő és a kaparással előállított regolitot felszívni, illetve esetlegesen maximum 50x25x25cm-es már szabadon lévő kőzetdarabokat begyűjteni. A marsjáró összesen 800kg kőzetminta tárolására és szállítására képes.

A minták tárolására két fajta tároló használható. Egy egyszer használatos, melyet a minták felhasználása után ki kell dobni, vagy egy speciális, vegyszeres tisztítás után újrahasznosítható. Szükség van emelett olyan "tárolókra", amelyekbe nem kerülnek minták, csupán arról adnak tanubizonyságot, hogy a kinyerés során nem szennyeződött a többi minta.

A fúró egy robotkarra van rögzítve, mely számos szabadsági fokkal, vagy izülettel rendelkezik. Első a váll, amely minden irányban mozgatható egy megadott tartományban. Ezt követi a könyök, amely hajlítható és egy 360 fokban elforgatható csukló. Ennek a végén található egy fogó, melynek egyik ága a fúró, a másik pedig egy kamera, amely a fúrást segíti. Ehhez egy lámpa is tartozik, és normál fénytartományban működik. Emellett található rajta egy biztonsági kontaktszenzor, amely véd a berendezés meghibásodása ellen.

A marsjárón a fúrófejhez tartozó kamera mellett található még 4, ebből 2 előre, 2 pedig hátrafelé néz. Ezek kellő távolságban vannak egymástól ahhoz, hogy a mélységérzékelés működjön. Minden kamerához tartozik 1-1 lámpa, és ezek az infravörös tartományban is működnek.

Az összegyűjtött mintákat el lehet tárolni, és a bázisra szállítani, vagy akár már a marsjáróban fel lehet őket nyitni és alapvető vizsgálatokat végezni rajtuk.

A vizsgálatok fő célja a bolygó geológiájának és történelmének feltárása, múltbéli élhetőségének vizsgálata, és víz és sejtes élet nyomainak keresése.

\subsubsection{Egyéb szenzorok}
A marsjáró optimális külső és belső állapotainak ellenőrzéséhez szükség van a navigációtól független szenzorokra is.

Ezekhez szükségesek:
\begin{itemize}
  \item külső szenzorok: \begin{itemize}
      \item külső hőmérő
      \item napállást vizsgáló szenzor
      \item légnyomásmérő
    \end{itemize}
  \item belső szenzorok: \begin{itemize}
      \item marsjáró és szkafanderek töltöttség-mérője
      \item oxigénszint-mérő
      \item nyomásszint-mérő
      \item belső hőmérő (ezáltal ellenőrizve a megrendelő által elvárt belső hőt)
      \item kőzetminta mérleg (hogy ne lépjük túl a megrendelő által maximálisan elvártan szállított kőzetek tömegét)
    \end{itemize}
\end{itemize}

Ezek által megállapítható, hogy:
\begin{itemize}
  \item alkalmasak-e a külső körülmények a marsjáró épségben maradását tekintve
  \item alkalmasak-e a belső körülmények marsjáróban lévő emberek számára
\end{itemize}

TODO: 
ötletek:

- sensory
  - manages the sensors and tools used for scientific research
  - analysis the data from them
  - stores (redundantly)

\subsubsection{Meghajtás}

Marsjáró révén, a meghajtás is fontos kérdés, TODO: meghajtásos kolléga

Ötletek:

- mobility

  - the driving mechanism e.g. motors, enginges

  - the (micro)controllers for them

Komponensek:

\begin{itemize}
    \item Kerék:
        \begin{itemize}
            \item Levegőmentes kerék
            \item Alumíniumból kell készíteni
            \item A tapadás érdekében kapcsokkal kell rendelkeznie
            \item A rugós tapadásért titánium külőket kell bele építeni
            \item Kerekek száma:
                \begin{itemize}
                    \item TODO: Össz hány darab?
                    \item TODO: Hány kerék meghajtású?
                    \item TODO: Hány pótkerék szükséges?
                \end{itemize}
        \end{itemize}
    \item Futómű:
        \begin{itemize}
            \item TODO: Anyaga?
            \item TODO: Terhetősége?
        \end{itemize}
    \item Motor:
        \begin{itemize}
            \item TODO: ?
        \end{itemize}
    \item Tank:
        \begin{itemize}
            \item Méretének akkorának kell, hogy 200 km üzemanyag elférjen benne
            \item Két azonos méretű üzemanyag cellából álljon
            \item A két üzemanyag cellából párhuzamosan fogyasszon üzemanyagot
            \item Az üzemanyag cellák legyenek képesek önállóan is ellátni az eszközt
            \item Feltölthető legyen a bázisról
            \item Legyen egy rendszer ami méri a tank az üzemanyagcellák töltöttségét
            \item Ez a rendszer értesítse a vezetőt, amikor a tank töltöttség félig van
            \item Cellák fala olyan anyagból készüljön, amelyet nem rongál az üzemanyag
            \item TODO: Cserélhetőnek kell-e lenni-e az üzemanyag celláknak?
        \end{itemize}
    \item Üzemanyag:
        \begin{itemize}
            \item TODO: ?
        \end{itemize}
    \item Közlekedés:
        \begin{itemize}
            \item El kell érnie a  25 km/h sebességet
            \item Képesnek kell lennie 20 nap alatt 20 km-t megtenni
            \item TODO: Sebesség fokozatok?
            \item TODO: Bevehető meredekség hajlásszöge?
            \item TODO: Szükséges-e valamilyen speciális manőver, például oldalazva menés?
        \end{itemize}
\end{itemize}


Becslések és gyártás:

\begin{itemize}
    \item Kerék:
        \begin{itemize}
            \item Beszerez: TODO: 
                \begin{itemize}
                    \item Titánium küllők: Total Titan Titanium
                    \item Aluminum: Cosmo Aluminum
                \end{itemize}
            \item Idő becslés: 2 hónap
            \item Pénz becslés: 450 AK
        \end{itemize}
    \item Tank:
        \begin{itemize}
            \item Beszerez: TODO: 
            \item Idő becslés: 1 hónap
            \item Pénz becslés: 100 AK
        \end{itemize}
    \item Motor:
        \begin{itemize}
            \item Beszerez: TODO: 
            \item Előfeltétel: 
                \begin{itemize}
                    \item Tank
                \end{itemize}
            \item Idő becslés: 1 év
            \item Pénz becslés: 10000 AK
        \end{itemize}
    \item Futómű:
        \begin{itemize}
            \item Beszerez: TODO: 
            \item Előfeltétel: 
                \begin{itemize}
                    \item Motor
                    \item Kerék
                \end{itemize}
            \item Idő becslés: 2 év
            \item Pénz becslés: 5500 AK
        \end{itemize}
    \item Összesített:
        \begin{itemize}
            \item Idő becslés: 3 év, 3 hónap
            \item Pénz becslés: 16050 AK
        \end{itemize}
\end{itemize}

\subsubsection{Létfentartás}

Természetesen az emberi személyzet végett, a létfenntartó rendszerek elengedhetetlenek. XYZ típusú levegő szűrő és széndioxid kivonó rendszerrel és XYZ típusú vízújrahasznosító rendszerrel fogjuk felszerelni. Az utóbbi képes a levegő páratartalmát kivonni, illetve vizeletet tisztítani, így végül fogyasztásra alkalmas ivóvizet kapunk.

Az asztronautáknak a megfelelő hőmérséklet biztosítására az XYZ típusú fűtő berendezést szereljük, mivel a marsi hőmörséklet nem haladja meg a 21 Celsius fokot, így fűtés biztosítása elegendő.

Forrás: https://www.weather.gov/fsd/mars
TODO: cite

\subsubsection{Telekommunikáció}

A telekommunkációhoz XYZ típusú, PRZ paraméterű antennákat és XYZ típusú jelfeldolgozó egységeket fogunk használni.

TODO: kérdés, bázissal kommunikációs frekvencia

\subsubsection{Szkafander/űrruha és a mars járó kapcsolata}
\begin{itemize}

  \item A bázissal megegyezö módon töltjük a szkafander elektromos akkumulátorát és oxigénpalackját is a roverben.
  \item A jármüvön egy fajta sürített o2 palack, amiböl a járón belül is keverjük a levegöt, és a palackot is tudjuk tölteni, ezt pedig a bázison töltjük újra. Biztonsági redundancia szempontból persze lehet több palack, de praktikus, ha csak egyfajta, és mindkét funkciót képes biztosítani.
  \item Töltöaljzat és oxigéncsap biztosítása.
  \item Szkafander töltö alkatrész, oxigén palack szükítö alkatrész.
  \item Ezek beszerelése a teljes elektromos és oxigén kivitelezési körök után kezdödhet, és átadásig elég befejezödnie.
  \item A tiszta oxigén légzése elött egy ú.n. elölégzésre van szükség, ami a nitrogént kiüríti a szervezetböl, így megelözve a sérüléseket. Ehhez szükséges vákuumkamra biztosítása. Javaslom a mars felé nyíló zsiliprendszer vákuumképzéssel felruházását.
  \item Oxigén nyomás kritikus csökkenés esetén, vagy kritikus áramcsökkenés esetén azonnali visszatérés megkövetelése. Fel nem töltött szkafanderek esetén töltés azonnali megkövetelése, vagy kötelezö visszatérés. A humán biztonság megköveteli, hogy mindig legyen töltött szkafander evakuációhoz.
\end{itemize}

\subsubsection{Energiaellátás}

TODO:

ötletek:
- power
  - managing the power in the rover
  - how electric power is flowing in the vechile

\section{Humán követelmények}
\begin{itemize}
  \item A marsjárón szkafander nélküli környezetet kell biztosítani. (Szobahömérséklet, levegö.)
  \item A marsjáron biztosítani kell a szkafanderek töltését ->bázissal egyezö módon.
  \item A küldetésekhez elegendö ételek tárolása és ivóvíz biztosítása is feltétel.
  Ezt a problémát két részre bontjuk, az első maga a tároló egység, amelyben az étel, és a szükséges ivóvíz van. A másik pedig a vizelet disztillálás, mellyel újrahasznosíthatjuk a vizeletet, izzadtságot.
  \begin{itemize}
    \item Vizelet disztilláló \\
    A NASA folyamatosan fejleszti azt az eszközt, amely a nemzetközi űrállomáson is megtalálható. A mai napig igyekeznek a vízszállítást a lehető legjobban csökkenteni. Egy ilyen disztillálóra lenne szükségünk tőlük, mely a legénység vízszükségleteinek 60\%-át tudja ellátni.
    Egy ilyennek a mérete 35-55 kilogram, térfogatban pedig 1 méter széles, 1,7 méter magas és fél méter a mélysége.
    Ennek elhelyezését a marsjáróban a egy toalett méretű helységben helyezném el. Ez a helység egy külön modul.
    Ez a gépezet naponta 9 kg vizeletet képes feldolgozni, ez kevesebb, mint amit 4 ember 16 óra alatt termel. A maradék vizeletüket el kell tárolni, hogy a későbbiekben felhasználhassák, akár a bázison, akár egy másik küldetésen. A vizelettárolást egy steril tartályban kell tárolni, amelynek hőmérséklete nem érheti el a 40 Celsius-fokot. Ennek megoldására, az újrahasznosítóval egy teremben kellene létrehozni egy toalett szerűséget, amely az említett tartályba gyűjti a vizeletet. Ezen tartálynak 380 literesnek kell lennie, hogy minden vizeletet megőrizzen, viszont hatalmas méretei miatt ezt 240 literesre tervezzük. A többi vizelet elveszhet. Ennek beszerzéséhez 400-500 euróra van szükség.
    \item Tároló egység \\
    A tároló raktár egy külön helység lesz a marsjárón, melyben különleges polcok vannak, amelyeknek az ételt tartalmazó részein a hőmérséklet állítható. Emelett olyan módon tartalmazza az ételt, hogy az a marsjáró különböző szögű mozgásaikor se essen le a polcról. Maga a moduláris terem beszerzése/beszerelése 4500-5000 euró. Ezen terem méretei: 2m belmagasság, 3 méter széles és 2 méter mély. A polcoknak együtt 140 kilogramm étel eltárolását kell biztosítani. Ezen polcok a teremben alul és felül is csatlakozzanak a felülettel. Egy ilyen polcnak az előállítása és beszerelésének költsége 350-400 euró, mivel négy ilyen polc van, ezért 1600 euró. \\
    A teremben szükség van még egy nyílásra, amely a 80 literes ivóvíz tartályhoz csatlakozik, ez a víz 40\% annak amelyet 4 ember megiszik 20 nap alatt. A többi víz a vizelet újrahasznosítással állítódik elő. A vizes tartály beszerzése és beszerelése eggyüttesen 3000-3500 euróba kerül.  
  \end{itemize}  
  \item Alapvetö higiénés és rekreációs szükségletek biztosítása feltétel.
\end{itemize}
\section{Kutatási követelmények}
\begin{itemize}
  \item A marsjáróra mintavételezéshez szükséges egy fúrót felszerelni.
  \item A begyüjött mintaásványokon kisebb laborvizsgálatokat el kell tudni végezni, mint pl. centrifugálás.
  \item Z köbméter ásvány ->bázisra való visszaszállítását kell biztosítani.
\end{itemize}
\section{Biztonsági követelmények}

A biztonságos misszióhoz elengedhetetlen a jármű kockázat analaízise, mely során annak alrendszereit feltérképezzük és a kritikusakat meghatározzuk. 

\subsection{Kockázat analízis}



A kockázat analízis célja, hogy a \ref{subsystems} fejezetben részletezett alrendszerek közül melyek a kritkus rendszerek és mennyire fontosak. A legnyílvánvalóbb az energia ellátó alrendszer, hiszen ha nincsen enerigai hiába van létfentartó rendszer vagy bármi más. Ha nincsen energia, akkor nem várható el egyik alrendszertől, vagy tartalékrendszertől sem, hogy működjön.

TODO: honna van energia, milyen energia hordozó, aztán tartalék akkumlátor, esetleg dízel generátor

Miután a járműnek van redundás energiaforrása, a létfenntartó rendszert fontos megvizsgálni. Az tényként kezelendő, hogy a fedélzeten utazó űrhajósokat életben kell tartni, így ezen rendszereknek is szükséges tartalékot képezni, legalább a levegő- és vízkezelés illetve a fűtés szintjén. 

\textit{Megjegyzés: A rover tartalékrendszerein túl az utasok bármikor felvehetik a szkafander sisakjukat, így akár egy harmadfokú renduncacnia is elérhető.}

A fent említett rendszereken túl, a harmadik a sorban a telekommunkációs rendszerek, hogy a kutatók tudjanak segítséget kérni a bázistól, vagy a másik rovertól amennyiben szükséges.

TODO: rendszerek megvalósítása







\begin{itemize}
  \item Az esetleges katasztrófaesemények esetén is biztosítani kell a személyzet ->bázisra való visszajutását.
\end{itemize}
\section{Utókövetelmények}
\begin{itemize}
  \item A marsjáró leszerelése és kezelése mint hulladék.
  \item A marsjáró által okozott esetleges természeti károk helyreállítási terve.
\end{itemize}

\chapter{Szójegyzék}
\section{Marsi fogalmak}
\begin{itemize}
  \item A marsi gravitáció x.
  \item A marsi év y. A fentiekben mi földi évben számolunk.
  \item A mars átlaghömérséklete z. k és j között ingadozik.
  \item A marsi UV l. Ez a földihez képest f/l, ami az eszközök élettartamát befolyásolja.
\end{itemize}

\section{További fogalmak}

\begin{itemize}
  \item \textbf{fogalom}: Lorem ipsom dolor sit amet TODO:
  \item \textbf{regolit}: Szilárd kérgű égitesteken a felszíni törmelékes, általában laza szerkezetű kőzetrétege. A talaj is a regolitnak nevezhető, de regolitról leginkább akkor beszélünk, ha a regolitréteg nem talajosodott.
  \item \textbf{geológia}: földtan
  \item \textbf{hőpárologtatás}: a kiásott lyuk mélyére egy hevítőrudat juttatnak, a hő hatására a kőzetben reakciók hajtódnak végre, anyagok válnak ki, amelyek vizsgálatával fontos információhoz juthatunk
\end{itemize}








\chapter{Overall Description TODO: REMOVE or REDESIGN}

\textbf{Megjegyzések, tanácsok a feladatokkal kapcsolatban} 
\begin{itemize}
    \item A rövid feladatkiírások csupán egy ügyfél elsokörös, magas szintugondolatait mutatják arról, hogy mit szeretne, nem fedik le a részleteket, speciális eseteket, tartalmazhatnak ellentmondásokat...stb.
    Ellentmondások keresése
    \item alacsonyabb szintre fordítás
    \item részletek
    \item speciális esetek (pl. marsi vagy földi év?)
    \item Puli marsjáró projekt utánajárás
    \item A feladat része a feladat pontosítása, megbeszélése az ügyféllel. Adott esetben szükséges az ügyfél figyelmét felhívni arra, hogy valamire nem gondolt  ellentmondás van a kéréseiben  kérése egyéb okok miatt nem megvalósítható    
    valamire nem gondolt kérése egyéb okok miatt nem megvalósítható
    \item A feladat része az is, hogy egy funkció megvalósítására alternatívákat kínálunk    
    alternatívák
    \item Fontos, hogy az egyes döntéseket az ügyféllel egyeztetve, vele egyetértésben kell meghozni
    kérdések előkészítése ügyfélnek
    \item Fontos, hogy az egyes döntéseket az ügyféllel egyeztetve, vele egyetértésben kell meghozni
    kérdések előkészítése ügyfélnek
    \item A feladatok olyanok, hogy 3-3 csapat 1-1 közös célt valósít meg; emiatt az egyeztetés a csapatok között a közös részek tekintetében elkerülhetetlen
    kapcsolat keresése Marsbázis projekttel
    \item A feladatok olyanok, hogy 3-3 csapat 1-1 közös célt valósít meg; emiatt az egyeztetés a csapatok között a közös részek tekintetében elkerülhetetlen
    \item Az egyeztetésnél a csapat egységesen lépjen fel az ügyféllel való kommunikációban. Kérdezhetnek többen is, de csapaton belüli esetleges kommunikációs hiányosságokat, inkonzisztenciákat, törésvonalakat ne mutassuk az ügyfélnek.
    5.a + performansz előkészítése, elpróbálása (legyen legalább 10-12 perc aktív tervünk, és hagyjunk helyet az ügyfél válaszára, ha nem tudunk csak 7 percnyi kérdést felvetni, akkor marad következő alkalomra  az ügyfél pontosításból már invalidálódhatnak eddigi kérdéseink)
    kérjük meg, hogy felvétel készülhessen, amiből scriptet vezessünk fel ide
    ha ez nem lehetséges, akkor valaki pontosan gyorsan gépírja, jegyzetelje le
    \item TODO: ??

\end{itemize}                  









\textbf{Mars-járó jármű}

Készítsen specifikációt 2 Mars-járó (gravitacio, guruló, hegymászó, repülö, vagy lábakon járó?) járműre az alábbiakat figyelembe véve:

\begin{itemize}
    \item A jármű élettartama minimum 5 év (marsi vagy földi év?) legyen a Marsi körülmények (utánajárás hogy mivel számoljunk) között
    \item ?? TODO?
\end{itemize}

A járműveknek maximum (minimum van? önjáró képesség igény-e?) 4 ember ellátását (oxigén, élelmezés, higiéniai szükségletek) kell biztosítaniuk a Mars-járó missziók időtartamára (milyen hosszú 1-1 misszió, mennyi idö telik el két misszió között karbantartási és ellátmánypótlási feladatokra)
• A jármű élettartama minimum 5 év (marsi vagy földi év?) legyen a Marsi körülmények (utánajárás hogy mivel számoljunk) között
• A járműnek lehetőséget kell biztosítania az űrhajósok kiszállására és beszállására
közvetlenül a Marsra vagy a marsi bázisra (bázissal konzultálni)
• A jármű képes legyen a misszióhoz szükséges segédfelszerelések szállítására, vizsgálatok
elvégzésére, valamint a begyűjtött kőzetminták szállítására
• Az eszközhöz csatlakozzon egy repülni képes, felderítő drón
• A járműnek képesnek kell lennie kommunikálnia a Marsbázissal és a Mars körül keringő
átjátszó műholdakkal valamint űrállomással
• Végezzen kockázatanalízist és a szükséges alrendszereknél alkalmazzon redundanciát
• A főbb alrendszereknek javíthatónak és cserélhetőnek kell lennie az ellátmány bázis
alkatrészei alapján

Todok

todo: where to collaborate? 
Robi: megcsinálom a pinned postot, a gyűjtéshez

Potenciális kommunikációs ember: Robi


Heninek: 
töltőállomások a bázison
Kiszállás hogy legyen?
Kommunikacio
Kérdések

0. kérdés: van-e komunikáció az órán kívül?
0.1: ha igen, akkor az csak 1 ember legyen-e?
0.2: az órán is csak 1 ember kommunikáljon az ügyféllel? 
bemehetünk-e a többi mars konzultációra megfigyelöként?
csatlakozhatnak-e a nem élöben teamsen a távoli tagok?
készíthetünk-e felvételt pl. teamssel?
ha nem, akkor jegyzökönyvezni kell -> legalább 2 fö kell a személyesre
backup: Józsi
Van-e igény arra, hogy miképpen “járjon” a Mars-járó?
Van-e minimum személyzet igény- legyen-e ön-távmüködö?
Marsi vagy földi év? Földi
Milyen hosszú 1-1 misszió, mennyi idö telik el két misszió között karbantartási és ellátmánypótlási feladatokra? (Ellatmanyurhajo csapat) 20 Nap, fél nap
Öntöltés requirement-e?/Töltöállomások?(Bázissal kommunikálni.)
Kell-e mennie nagyon hidegben? szélöseg -60 fok, de minden körülményt ki kell birni
De ha sem, akkor kell-e fütést biztosítani? 22 fok
Vagy a szkafander füt-e? igen
Azt kell-e tölteni? igen
Pótolni?
Milyen javíthatósági képesség legyen? Pl. bázis, másik járó, önmaga? potalkatreszek, zsiliprendszer
Hány ajtós legyen? Zsilip es bazis ajtó, ezek mas magasságba vannak
Hogy képzeli el a felszerelések szállításat? 20 cmes kockák, kell tudni fúrni
Milyen vizsgalatokra tart igenyt?kell furofej, hoparologtatasos vizsgalat, levágni megfelelő nagyságú 
mekkora kozetek? oklomnyi/olympus. 800kg, masfel tonna
Csak a Marsjáró irányitsa-e a dront? biztonsagi funkcioknak alapvető vezérlés, de ember is iranyítja Mekkora hataskor? 100m? 4,5 km és autómatikusan visszatalál
Milyen alrendszereknek kell biztositanunk a redundanciajat? Milyen mertekben? Marsi vihart is tul kell-e elnie? igen, de nem kell alatta mintat venni
 ellatmanyurhajo rank is szallhat, vagy mindig a bazisra?
Budget? 
Qa
Functional
And nonfunctional
Design
Mit tudunk a megrendelőről, piaci vagy állami? 
Milyen külső követelményeknek kell megfelelnünk? Todo: van-e személyszállítási ISO, ha igen mi, ennek meg kell-e felelni? 
\section{Product Perspective}
\section{User Classes and Characteristics}
% add other chapters and sections to suit
\end{document}
