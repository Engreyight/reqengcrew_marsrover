\documentclass{report}
\usepackage{listings}
\usepackage{underscore}
\usepackage{todonotes}
\usepackage[bookmarks=true]{hyperref}
\hypersetup{
    bookmarks=false,    % show bookmarks bar?
    pdftitle={Software Requirement Specification},    % title
    pdfauthor={Yiannis Lazarides},                     % author
    pdfsubject={TeX and LaTeX},                        % subject of the document
    pdfkeywords={TeX, LaTeX, graphics, images}, % list of keywords
    colorlinks=true,       % false: boxed links; true: colored links
    linkcolor=blue,       % color of internal links
    citecolor=black,       % color of links to bibliography
    filecolor=black,        % color of file links
    urlcolor=purple,        % color of external links
    linktoc=page            % only page is linked
}%



\title{%


\Huge{KÖVETELMÉNYELEMZÉS}\\
\vspace{2cm}

Marsjáró\\
\vspace{2cm}

projekthez\\
\vspace{2cm}

\LARGE{Verzió 1.0\\}
\vspace{2cm}

\vspace{2cm}
Készítették: [tagok]\\ % todo!
\vfill

}
\date{}
\usepackage{hyperref}
\begin{document}
\listoftodos
\maketitle
\tableofcontents


\chapter{Revision History}
\todo{Revíziós tábla hozzáadása}

\chapter{Projekt bemutatása}
\todo{Néhány gondolat a projekt bemutatására}

\chapter{Követelmények}
\section{Követelmények felépítése}
\section*{Megvalósítás lépései}
\section*{...}
\todo{További szerkezet megadása}

\chapter{Szójegyzék}

\section{References (??)}





\chapter*{Feladatkörök (TODO: REMOVE???)}
\begin{itemize}
    \item Heni: kapcsolat keresése Marsbázis projekttel
    \item Tihi: kapcsolat keresése Ellátmányürhajóval
    \item Robi: elsődleges kapcsolattartó ember
    
\end{itemize}


\chapter{Overall Description TODO: REMOVE or REDESIGN}

\textbf{Megjegyzések, tanácsok a feladatokkal kapcsolatban} 
\begin{itemize}
    \item A rövid feladatkiírások csupán egy ügyfél elsokörös, magas szintugondolatait mutatják arról, hogy mit szeretne, nem fedik le a részleteket, speciális eseteket, tartalmazhatnak ellentmondásokat...stb.
    Ellentmondások keresése
    \item alacsonyabb szintre fordítás
    \item részletek
    \item speciális esetek (pl. marsi vagy földi év?)
    \item Puli marsjáró projekt utánajárás
    \item A feladat része a feladat pontosítása, megbeszélése az ügyféllel. Adott esetben szükséges az ügyfél figyelmét felhívni arra, hogy valamire nem gondolt  ellentmondás van a kéréseiben  kérése egyéb okok miatt nem megvalósítható    
    valamire nem gondolt kérése egyéb okok miatt nem megvalósítható
    \item A feladat része az is, hogy egy funkció megvalósítására alternatívákat kínálunk    
    alternatívák
    \item Fontos, hogy az egyes döntéseket az ügyféllel egyeztetve, vele egyetértésben kell meghozni
    kérdések előkészítése ügyfélnek
    \item Fontos, hogy az egyes döntéseket az ügyféllel egyeztetve, vele egyetértésben kell meghozni
    kérdések előkészítése ügyfélnek
    \item A feladatok olyanok, hogy 3-3 csapat 1-1 közös célt valósít meg; emiatt az egyeztetés a csapatok között a közös részek tekintetében elkerülhetetlen
    kapcsolat keresése Marsbázis projekttel
    \item A feladatok olyanok, hogy 3-3 csapat 1-1 közös célt valósít meg; emiatt az egyeztetés a csapatok között a közös részek tekintetében elkerülhetetlen
    \item Az egyeztetésnél a csapat egységesen lépjen fel az ügyféllel való kommunikációban. Kérdezhetnek többen is, de csapaton belüli esetleges kommunikációs hiányosságokat, inkonzisztenciákat, törésvonalakat ne mutassuk az ügyfélnek.
    5.a + performansz előkészítése, elpróbálása (legyen legalább 10-12 perc aktív tervünk, és hagyjunk helyet az ügyfél válaszára, ha nem tudunk csak 7 percnyi kérdést felvetni, akkor marad következő alkalomra  az ügyfél pontosításból már invalidálódhatnak eddigi kérdéseink)
    kérjük meg, hogy felvétel készülhessen, amiből scriptet vezessünk fel ide
    ha ez nem lehetséges, akkor valaki pontosan gyorsan gépírja, jegyzetelje le
    \item 

\end{itemize}                  









\textbf{Mars-járó jármű}

Készítsen specifikációt 2 Mars-járó (gravitacio, guruló, hegymászó, repülö, vagy lábakon járó?) járműre az alábbiakat figyelembe véve:

\begin{itemize}
    \item A jármű élettartama minimum 5 év (marsi vagy földi év?) legyen a Marsi körülmények (utánajárás hogy mivel számoljunk) között
    \item 
\end{itemize}

A járműveknek maximum (minimum van? önjáró képesség igény-e?) 4 ember ellátását (oxigén, élelmezés, higiéniai szükségletek) kell biztosítaniuk a Mars-járó missziók időtartamára (milyen hosszú 1-1 misszió, mennyi idö telik el két misszió között karbantartási és ellátmánypótlási feladatokra)
• A jármű élettartama minimum 5 év (marsi vagy földi év?) legyen a Marsi körülmények (utánajárás hogy mivel számoljunk) között
• A járműnek lehetőséget kell biztosítania az űrhajósok kiszállására és beszállására
közvetlenül a Marsra vagy a marsi bázisra (bázissal konzultálni)
• A jármű képes legyen a misszióhoz szükséges segédfelszerelések szállítására, vizsgálatok
elvégzésére, valamint a begyűjtött kőzetminták szállítására
• Az eszközhöz csatlakozzon egy repülni képes, felderítő drón
• A járműnek képesnek kell lennie kommunikálnia a Marsbázissal és a Mars körül keringő
átjátszó műholdakkal valamint űrállomással
• Végezzen kockázatanalízist és a szükséges alrendszereknél alkalmazzon redundanciát
• A főbb alrendszereknek javíthatónak és cserélhetőnek kell lennie az ellátmány bázis
alkatrészei alapján

Todok

todo: where to collaborate? 
Robi: megcsinálom a pinned postot, a gyűjtéshez

Potenciális kommunikációs ember: Robi


Heninek: 
töltőállomások a bázison
Kiszállás hogy legyen?
Kommunikacio
Kérdések

0. kérdés: van-e komunikáció az órán kívül?
0.1: ha igen, akkor az csak 1 ember legyen-e?
0.2: az órán is csak 1 ember kommunikáljon az ügyféllel? 
bemehetünk-e a többi mars konzultációra megfigyelöként?
csatlakozhatnak-e a nem élöben teamsen a távoli tagok?
készíthetünk-e felvételt pl. teamssel?
ha nem, akkor jegyzökönyvezni kell -> legalább 2 fö kell a személyesre
backup: Józsi
Van-e igény arra, hogy miképpen “járjon” a Mars-járó?
Van-e minimum személyzet igény- legyen-e ön-távmüködö?
Marsi vagy földi év? Földi
Milyen hosszú 1-1 misszió, mennyi idö telik el két misszió között karbantartási és ellátmánypótlási feladatokra? (Ellatmanyurhajo csapat) 20 Nap, fél nap
Öntöltés requirement-e?/Töltöállomások?(Bázissal kommunikálni.)
Kell-e mennie nagyon hidegben? szélöseg -60 fok, de minden körülményt ki kell birni
De ha sem, akkor kell-e fütést biztosítani? 22 fok
Vagy a szkafander füt-e? igen
Azt kell-e tölteni? igen
Pótolni?
Milyen javíthatósági képesség legyen? Pl. bázis, másik járó, önmaga? potalkatreszek, zsiliprendszer
Hány ajtós legyen? Zsilip es bazis ajtó, ezek mas magasságba vannak
Hogy képzeli el a felszerelések szállításat? 20 cmes kockák, kell tudni fúrni
Milyen vizsgalatokra tart igenyt?kell furofej, hoparologtatasos vizsgalat, levágni megfelelő nagyságú 
mekkora kozetek? oklomnyi/olympus. 800kg, masfel tonna
Csak a Marsjáró irányitsa-e a dront? biztonsagi funkcioknak alapvető vezérlés, de ember is iranyítja Mekkora hataskor? 100m? 4,5 km és autómatikusan visszatalál
Milyen alrendszereknek kell biztositanunk a redundanciajat? Milyen mertekben? Marsi vihart is tul kell-e elnie? igen, de nem kell alatta mintat venni
 ellatmanyurhajo rank is szallhat, vagy mindig a bazisra?
Budget? 
Qa
Functional
And nonfunctional
Design
Mit tudunk a megrendelőről, piaci vagy állami? 
Milyen külső követelményeknek kell megfelelnünk? Todo: van-e személyszállítási ISO, ha igen mi, ennek meg kell-e felelni? 
\section{Product Perspective}
\section{User Classes and Characteristics}
% add other chapters and sections to suit
\end{document}
